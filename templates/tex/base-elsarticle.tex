\documentclass[twocolumn,10pt]{elsarticle}

\usepackage[columnwise, switch]{lineno}
\usepackage{hyperref}
\usepackage{amsfonts,amsmath,amssymb}
\usepackage{bm}
\usepackage{etoolbox}
\AtBeginEnvironment{frontmatter}{\onecolumn}
\AfterEndEnvironment{frontmatter}{\twocolumn}

\journal{Combustion and Flame}

%%%%%%%%%%%%%%%%%%%%%%%
%% Elsevier bibliography styles
%%%%%%%%%%%%%%%%%%%%%%%
%% To change the style, put a % in front of the second line of the current style and
%% remove the % from the second line of the style you would like to use.
%%%%%%%%%%%%%%%%%%%%%%%

%% Numbered
%\bibliographystyle{model1-num-names}

%% Numbered without titles
%\bibliographystyle{model1a-num-names}

%% Harvard
%\bibliographystyle{model2-names.bst}\biboptions{authoryear}

%% Vancouver numbered
%\usepackage{numcompress}\bibliographystyle{model3-num-names}

%% Vancouver name/year
%\usepackage{numcompress}\bibliographystyle{model4-names}\biboptions{authoryear}

%% APA style
%\bibliographystyle{model5-names}\biboptions{authoryear}

%% AMA style
%\usepackage{numcompress}\bibliographystyle{model6-num-names}

%% `Elsevier LaTeX' style
\bibliographystyle{elsarticle-num}
%%%%%%%%%%%%%%%%%%%%%%%

\begin{document}

\begin{frontmatter}

\title{Awsome title}

\author[IFS, SoE]{Akira Tsunoda\corref{mycorrespondingauthor}}
\cortext[mycorrespondingauthor]{Corresponding author}
\ead{tsunoda@edyn.ifs.tohoku.ac.jp}

\author[IFS]{Youhi Morii}

\address[IFS]{Institue of Fluid Science, Tohoku University,\\ 2-1-1 Katahira, Aoba-ku, Sendai, Miyagi, 9808577 Japan}
\address[SoE]{School of Engineering, Tohoku University,\\ 6-6 Aoba, Aramaki, Aoba-ku, Sendai, Miyagi, 9808579 Japan}

\begin{abstract}
This template helps you to create a properly formatted \LaTeX\ manuscript.
\end{abstract}

\begin{keyword}
Combustion \sep Flame \sep Stretch \sep Flame ball
\end{keyword}

\end{frontmatter}

\linenumbers
\twocolumn

\section{Introduction}

\section{Methods}

\section{Results and discussion}

\section{Bibliography styles}

\section*{References}

%\bibliography{mybibfile}

\end{document}
